\section{Outer-Loop Controller}
\label{sec:outer_loop}

The outer-loop controller calculates desired attitude of the quadrotor, based on a given desired position and the quadrotor's current position. As a method of controlling the position of the quadrotor, PID control is applied. In this section, the outer-loop controller will be introduced. First, dynamics with respect to the quadrotor's position is described. Then, the position control of the quadrotor is stated.

\subsection{Dynamics Description}

A quadrotor is accelerated by thrust generated by its four rotors. Let \( {\boldsymbol F} = (\overline{X}, \overline{Y}, \overline{Z})\) be the quadrotor's thrust specified in the inertial frame. Then, the equation of motion is given as,
\begin{equation}
\label{eq:newton}
\begin{aligned}
m \ddot{{\boldsymbol{r}}} = \boldsymbol{F} + m \boldsymbol{g}\\
\end{aligned}
\end{equation}
where \(m\) is the mass of the quadrotor, \({\boldsymbol r} = (x, y, z)\) is the position of the quadrotor in the inertial frame, and \(\boldsymbol{g} = (0, 0, -g)\) is the gravitational acceleration.

Since the rotation axis of each rotor is fixed perpendicularly to the body, thrust can be generated only in the perpendicular direction of the body frame. Therefore, thrust can be computed by a rotation transformation from the body frame into the inertial frame. From equation (\ref{eq:transform_matrix}), thrust vector in the inertial frame is given as,
\begin{equation}
\begin{aligned}
{\boldsymbol F}
& = {^I_B}{R} 
\begin{bmatrix}
0\\
0\\
F
\end{bmatrix}\\
& =
\begin{bmatrix}
\cos{\theta} \cos{\psi} & \sin{\phi} \sin{\theta} \cos{\psi} - \cos{\phi}\sin{\psi} & \cos{\phi} \sin{\theta} \cos{\psi} + \sin{\phi} \sin{\psi}\\
\cos{\theta} \sin{\psi} & \sin{\phi} \sin{\theta} \sin{\psi} + \cos{\phi}\cos{\psi} & \cos{\phi} \sin{\theta} \sin{\psi} - \sin{\phi} \cos{\psi}\\
-\sin{\theta} &  \sin{\phi} \cos{\theta} & \cos{\phi} \cos{\theta}
\end{bmatrix}
\begin{bmatrix}
0\\
0\\
F
\end{bmatrix}\\
& =
F
\begin{bmatrix}
\cos{\phi} \sin{\theta} \cos{\psi} + \sin{\phi} \sin{\psi}\\
\cos{\phi} \sin{\theta} \sin{\psi} - \sin{\phi} \cos{\psi}\\
\cos{\phi} \cos{\theta}\\
\end{bmatrix}
\end{aligned}
\end{equation}
where \( F = \sqrt{{\overline{X}}^2 + {\overline{Y}}^2 + {\overline{Z}}^2}\) is the magnitude of the thrust, and \( {\boldsymbol \eta} = (\phi, \theta, \psi) \) is the attitude (roll, pitch, and yaw, respectively) of the quadrotor. Then, the thrust along each axis of the inertial frame is given as,
\begin{equation}
\label{eq:thrust_vector}
\begin{aligned}
{\overline X} &=  F (\cos{\phi} \sin{\theta} \cos{\psi} + \sin{\phi} \sin{\psi})\\
{\overline Y} &= F (\cos{\phi} \sin{\theta} \sin{\psi} - \sin{\phi} \cos{\psi})\\
{\overline Z} &=  F \cos{\phi} \cos{\theta}\\
\end{aligned}
\end{equation}
Therefore, from Equations (\ref{eq:newton}) and (\ref{eq:thrust_vector}), the acceleration of the quadrotor \({(\ddot{x}}, {\ddot{y}}, {\ddot{z}})\) is computed as,
\begin{equation}
\begin{aligned}
{{\ddot x} } & =  {F \over m} (\cos{\phi} \sin{\theta} \cos{\psi} + \sin{\phi} \sin{\psi})\\
{{\ddot y} } & =  {F \over m} (\cos{\phi} \sin{\theta} \sin{\psi} - \sin{\phi} \cos{\psi})\\
{{\ddot z} } & =  {F \over m} \cos{\phi} \cos{\theta} -g\\
\end{aligned}
\end{equation}

The sensors are fixed on the quadrotor, and therefore, it is necessary to formulate the relation of the quadrotor motion in the body frame and the inertial frame. Let \({\boldsymbol v}_B = (u, v, w)\) be the quadrotor's velocity in the body frame and \({\boldsymbol \omega} = (p, q, r)\) be the quadrotor's angular velocity. From the Coriolis effect, the equation of motion is given with respect to \( {\boldsymbol v}_B \) as the below equation \cite{randal08}.
\begin{equation}
\label{eq:newton_gravity}
\begin{aligned}
m \ddot{{\boldsymbol{r}}} = m (\dot{{\boldsymbol v}}_B + {\boldsymbol \omega} \times {\boldsymbol v}_B)\\
\end{aligned}
\end{equation}
Hence, from Equation (\ref{eq:newton}) the equation of the quadrotor's thrust \({\boldsymbol F}\) is written as,
\begin{equation}
\begin{aligned}
\boldsymbol{F} = m (\dot{{\boldsymbol v}}_B + {\boldsymbol \omega} \times {\boldsymbol v}_B- \boldsymbol{g})\\
\end{aligned}
\end{equation}

\subsection{Computation of Desired Velocity}
In order to control the quadrotor's position, the position controller sets desired velocity \({\boldsymbol v}_d = ({v_x}_d, {v_y}_d, {v_z}_d) \) in the inertial frame, with respect to the error of desired position \( {\boldsymbol{r}}_d = ({x_d} , {y_d} , {z_d} ) \) and observed position \( {{\boldsymbol{r}}}  = ({{x}} , {{y}} , {{z}} )\). The desired velocity is generated by P control. Desired control along each axis of the inertial frame is given as,
\begin{equation}
\begin{aligned}
{\boldsymbol v}_d = K_{pos} ( {\boldsymbol r}_d - {\boldsymbol r})\\
\end{aligned}
\end{equation}
where \( K_{pos} \) is a diagonal gain matrix of the controller. In order to prevent abnormal performance of the quadrotor, we set an upper limits of desired velocity. 

\subsection{Computation of Desired Thrust and Attitude}

In order to control the quadrotor's velocity, PID control can be applied to the motion of each axis in the inertial frame. A PID position control is given as the below equation \cite{Morgan16}\cite{giri}.
\begin{equation}
\label{eq:original_pid}
\begin{aligned}
{\overline X}_d & = m \left( \ddot{x}_d + {{K_P}_x}({x_d} - { x} ) + {{K_D}_x}({\dot {x}_d} -\dot{ x} ) + {{K }_I}_x \int_{t_0}^{t}({{x_d} - { x} }) dt \right) \\
{\overline Y}_d & = m \left( \ddot{y}_d + {{K_P}_y}({y_d} - { y} ) + {{K_D}_y}({\dot {y}_d} -\dot{ y} ) + {{K }_I}_y \int_{t_0}^{t}({{y_d} - { y} }) dt \right) \\
{\overline Z}_d & = m \left( g + \ddot{z}_d  + {{K_P}_z}({z_d} - { z} ) + {{K_D}_z}({\dot {z}_d} -\dot{ z} ) + {{K }_I}_z \int_{t_0}^{t}({{z_d} - { z} }) dt \right)\\
\end{aligned}
\end{equation}
\({{K_{P}}_x}\), \({{K_{P}}_y}\), \({{K_{P}}_z}\) are the gains of the proportional terms, \({{K_{D}}_x}\), \({{K_{D}}_y}\), \({{K_{D}}_z}\) are gains of the differential terms, and \({{K_{I}}_x}\), \({{K_{I}}_y}\), \({{K_{I}}_z}\) are gains of the integral terms.

The Pixhawk autopilot has been developed to support both indoor and outdoor flight. However, outdoor flight puts priority on velocity control, and considering outdoor environments, the control law of Equation (\ref{eq:original_pid}) may not work properly without precise position measurement. Therefore, a PID controller in terms of the quadrotor velocity is used alternatively in the autopilot system. The desired thrust of the alternative PID controller is given as,\\
\begin{equation}
\begin{aligned}
\ddot{x}  & = {1 \over m} \left( {{k_P}_x}({v_x}_d - \dot{ x} ) + {{k_D}_x}({\dot {v}_{x,d}} -\ddot{ x} ) + {{k_I }_x} \int_{t_0}^{t}({{v_x}_d - \dot{ x} }) dt \right)\\
\ddot{y}  & = {1 \over m} \left( {{k_P}_y}({v_y}_d - \dot{ y} ) + {{k_D}_y}({\dot {v}_{y,d}} -\ddot{ y} ) + {{k_I }_y} \int_{t_0}^{t}({{v_y}_d - \dot{ y} }) dt \right)\\
\ddot{z}  & = {1 \over m} \left( {{k_P}_z}({v_z}_d - \dot{ z} ) + {{k_D}_z}({\dot {v}_{z,d}} -\ddot{ z} ) + {{k_I }_z} \int_{t_0}^{t}({{v_z}_d - \dot{ z} }) dt \right)\\
\end{aligned}
\end{equation}
where \({{k_{P}}_x}\), \({{k_{P}}_y}\), \({{k_{P}}_z}\) are gains of the proportional terms, \({{k_{D}}_x}\), \({{k_{D}}_y}\), \({{k_{D}}_z}\) are gains of the differential terms, and \({{k_{I}}_x}\), \({{k_{I}}_y}\), \({{k_{I}}_z}\) are gains of the integral terms. Then, from Equation (\ref{eq:newton}), the desired thrust \({\overline X}_d, {\overline Y}_d, {\overline Z}_d \) can be computed as,
\begin{equation}
\label{eq:desired_thrust_vector}
\begin{aligned}
{\overline X}_d & = {{k_P}_x}({v_x}_d - \dot{ x} ) + {{k_D}_x}({\dot {v_x}_d} -\ddot{ x} ) + {{k_I }_x} \int_{t_0}^{t}({{v_x}_d - \dot{ x} }) dt\\
{\overline Y}_d & = {{k_P}_y}({v_y}_d - \dot{ y} ) + {{k_D}_y}({\dot {v_y}_d} -\ddot{ y} ) + {{k_I }_y} \int_{t_0}^{t}({{v_y}_d - \dot{ y} }) dt\\
{\overline Z}_d & = m g + {{k_P}_z}({v_z}_d - \dot{ z} ) + {{k_D}_z}({\dot {v_z}_d} -\ddot{ z} ) + {{k_I }_z} \int_{t_0}^{t}({{v_z}_d - \dot{ z} }) dt\\
\end{aligned}
\end{equation}

In this research, Equation (\ref{eq:desired_thrust_vector}) is used for outer-loop control, but the control law of Equation of (\ref{eq:original_pid}) is more appropriate for indoor flight with the values of \((\dot{x}_d, \dot{ y}_d, \dot{z}_d)\) and \((\ddot{x}_d, \ddot{ y}_d, \ddot{z}_d)\). In the control of Equation (\ref{eq:desired_thrust_vector}), the gains of differential terms are set small so that the effect of the differential terms is comparatively ignorable. Since the desired velocity is set proportionally with the error of the quadrotor position, the control law becomes similar to PD control in terms of the quadrotor position.

As described above, the direction of the thrust is defined by the orientation of the quadrotor. From Equations (\ref{eq:thrust_vector}), desired roll \(\phi_d\), pitch \(\theta_d\), and yaw \(\psi_d\) are computed as,
\begin{equation}
\label{eq:desired_attitude}
\begin{aligned}
\phi_d = \arcsin {{{{\overline X}_d} \sin{\psi_d} - {{\overline Y}_d} \cos{\psi_d}} \over {F_d}}\\
\theta_d = \arctan {{{{\overline X}_d} \cos{\psi_d} + {{\overline Y}_d} \sin{\psi_d}} \over {{\overline Z}_d}}\\
\end{aligned}
\end{equation}
Also, the magnitude of desired thrust \(F_d\) is given as,\\
\begin{equation}
\begin{aligned}
F_d & = | {\boldsymbol F}_d | \\
& =  \sqrt{{{\overline X}_d}^2 + {{\overline Y}_d}^2 + {{\overline Z}_d}^2}\\
\end{aligned}
\end{equation}
%This implies that with any given yaw angle, the position of the quadrotor is controllable. Therefore, yaw angle is also controllable while holding the quadrotor's position.
\( {{\overline X}_d} \), \({{\overline Y}_d}\), and \({{\overline Z}_d}\) are computed from Equation (\ref{eq:desired_thrust_vector}). Therefore, we can find that the desired roll \(\phi_d\) and pitch \(\theta_d\) are determined by the given desired yaw \(\psi_d\). 
