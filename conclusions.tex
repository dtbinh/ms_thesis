\chapter{Conclusion}
\label{ch:conclusion}

In this research, a dynamic model-based nonlinear attitude control was developed. First, we developed a quadrotor system testbed that satisfies the requirements for potential future research. Then, we studied the dynamics of the quadrotor and developed a nonlinear inner-loop attitude controller based on the model. Further, by characterizing the motor dynamics and analyzing the propeller aerodynamics, we developed an advanced model of the quadrotor that allows more stable and agile control. Based on the model, we developed a nonlinear attitude controller and applied it to the quadrotor system.
 
The nonlinear control that has been developed in this research has an advantage of globally exponentially stability. Therefore, if a proper dynamic model for the quadrotor can be defined, the control law is expected to allow better performance of the quadrotor. In order to validate the maneuver advantage of the nonlinear controller, the quadrotor's performance was evaluated by both simulations and experiments. The flight performance of the quadrotor with the nonlinear attitude control is compared with the performance of a conventional PID attitude control, and it was verified that the nonlinear attitude control is superior in stability and agility, and therefore, effective for a quadrotor UAVs. By further tuning the gains of the nonlinear controller, we can expect to clearly establish the benefits of the proposed nonlinear control method.

However, in order to develop fully-autonomous system for the quadrotor, it is necessary to install position measurement on the quadrotor. Therefore, in Chapter \ref{ch:vision_based_control}, a computationally-efficient vision-based position estimator was designed. Since the position estimator is executable in real-time, it is expected to be used in the quadrotor system so that the quadrotor corrects accumulated position errors.

As for future works, synchronized with the nonlinear attitude control, the application of the motion planning and guidance is expected to further improve the quadrotor's performance \cite{motion_planning}\cite{Morgan16}\cite{Morgan14}. Especially for path planning of rapid maneuvering, high nonlinearity of an aerial vehicle is challenging which possibly exacerbates and restricts its maneuver \cite{Paranjape15}. However, the nonlinear attitude controller developed in this research is expected to guarantee stability and allow fast maneuver. Also, in this research, the motor dynamics is approximated by experiments, but precise modeling of a motor would increase the robustness of the inner-loop control as well. The nonlinear attitude inner-loop controller has an advantage of stability in terms of angular velocity as well as attitude, and therefore, the nonlinear quadrotor system is expected to be applied to a quadrotor's acrobatic flight that requires high response speed of angular velocity.